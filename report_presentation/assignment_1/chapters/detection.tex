\section{Detection}
The server, upon sending the goal to the robot, also initiates the process of reading the messages that the robot sends on the topic \texttt{/scan\_raw}.
On this topic, the robot transmits a list of distances obtained from a scan of the half space in front of it: the scan extends from an angle of \textbf{-1.9199 radians} to \textbf{1.9199 radians}, where \textbf{0 radians} represents the direction directly in front of the robot, resulting in a field of view slightly larger than \textbf{180 degrees}.\\

The server analyzes these distances and groups them: a group signifies a continuous surface, thus forming a new group whenever a distance significantly differs from the previous one. A continuous set of points within a group is defined as a profile, with each point in a profile not differing from its neighbors by more than \textbf{0.5 meters}.\\\\

Each profile undergoes scrutiny to ascertain whether it resembles a semi-circle, akin to what would be seen if a cylinder were present. The assessment of a profile's semi-circular nature involves several criteria:
\begin{itemize}
    \item The distance between the nearest point to the first and last points of the profile should be similar (difference $<$ \textbf{0.1}).
    \item The midpoint between the first and last points, presumed to be the circle's center, should share a similar distance with the nearest point.
    \item The radius (i.e., the distance from the center to one of the profile points) should approximate the expected radius of the cylinders being detected (approximately \textbf{0.17 meters}).
    \item If these conditions are met, a stronger but slower check is performed to further validate the profile. This additional check examines every point's distance from the center, ensuring similarity to the circle's radius.
\end{itemize}
Upon satisfying all conditions, the profile is labeled as a circle; otherwise, it is marked as unknown. The list of profiles identified as cylinders is saved and transmitted with the feedback. If the current list's number of elements is greater than or equal to the largest previously seen list, it replaces the previous record. This approach aids in updating the list, accommodating any changes in cylinder positions that may occur while the robot continues detecting.\\\\

Updating the larger list, even if the current one holds the same number of elements, ensures adaptation in case any of the cylinders are repositioned during the robot's detection process. If the robot re-observes all the cylinders simultaneously, it updates the list with the revised positions.\\
